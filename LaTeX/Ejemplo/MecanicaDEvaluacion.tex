\documentclass[12pt]{article}
%Preamble
\usepackage[spanish]{babel}
\usepackage[latin1]{inputenc}
		 \def\Centrar#1
		 {
			\begin{center}
			#1
		 	\end{center}
		 }
		
		 \newcount\EvidCountTUno\EvidCountTUno0
		
		 \def\EvidTipoUno#1{
		 \global\advance\EvidCountTUno by 1
		 			  {\large\bf Evidencia tipo 1 No. \the\EvidCountTUno
		 			  \\ #1}
		 }%end macro \EvidTipoUno
		
		 \newcount\EvidDTellez
	  	 	\EvidDTellez=0
		
		 \def\EvidenciaDTellez#1{
		 	\global\advance\EvidDTellez by 1
		 	Tellez P\'erez Jos�&#1\\\hline
		 }

		 \newcount\EvidDAcosta
	  	 	\EvidDAcosta=0
		
		 \def\EvidenciaDAcosta#1{
		 	\global\advance\EvidDAcosta by 1
		 	Acosta Tolentino Alan&#1\\\hline
		 }
		
\begin{document}
				\noindent\EvidTipoUno{Calcular superf\/icie de un circulo}\par
				
				\Centrar{
				\begin{tabular}{|l|p{1.5in}|}\hline
								 %Tellez P\'erez Jos�&Evidencia completa\\\hline
								 \EvidenciaDTellez{Evidencia completa}
								 %Acosta Tolentino Alan&OK\\\hline
								 %\EvidenciaDAcosta{OK}
		 		\end{tabular}
				}
				
                Ac� tendremos la continuaci�n de este documento
			  	\eject
				\noindent\EvidTipoUno{Un programa en Fortran}\par
				\Centrar{
				\begin{tabular}{|l|p{1.5in}|}\hline
								 %Tellez P\'erez Jos�&Evidencia completa\\\hline
								 \EvidenciaDTellez{Evidencia completa}
								 %Acosta Tolentino Alan&OK\\\hline
								 \EvidenciaDAcosta{OK}
		 		\end{tabular}
				}

				\noindent Resumen de las evidencias tipo 1 recibidas\par
				\Centrar{
		 		\begin{tabular}{|l|r|}\hline
			 	Tellez Perez Jose&\the\EvidDTellez\\\hline
			 	Acosta Tolentino Alan&\the\EvidDAcosta\\\hline
			 	\end{tabular}
			 	}

				\eject
				\noindent Ejemplo de una clase en C++
				\begin{verbatim}
				class Fecha{
					  int dia;
					  int mes;
					  int anio;
                public: Fecha(int,int,int);
                char *nombre_del_dia();/* "Lunes", "Martes", etc. */
		  		};

				\end{verbatim}

\end{document}
